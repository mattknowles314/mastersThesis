
\documentclass{beamer}

\usetheme{Berlin}
\usecolortheme{beaver}

\usepackage{graphicx} 
\usepackage{booktabs} 

\title[Improving DLS]{Improving Duckworth-Lewis: Statistical Methods for Resetting Score Targets in Limited-Overs Cricket} 

\author{Matthew Knowles} 
\institute[UoY] 
{
University of York \\ 
\medskip
\textit{mk1320@york.ac.uk} 
}
\date{\today} 

\begin{document}

\begin{frame}
\titlepage 
\end{frame}

\begin{frame}
\frametitle{Plan for the Talk} 
\tableofcontents 
\end{frame}

\section{Background}
\begin{frame}
\frametitle{Cricket}

\begin{itemize}
    \item 2 teams of 11 players. One team bats, the other fields. \\
    \pause
    \item Score measured in runs. Aim: Score as many as you can before losing 10 wickets. \\
    \pause
    \item Focus in this project is limited overs cricket. Games last 50 overs, which takes about 3 hours.
\end{itemize}

\end{frame}

\begin{frame}
\frametitle{Setting the scene: The need for score resetting}

\begin{itemize}
    \item Cricket is very sensitive to external factors, such as rain and daylight. \\
    \item If it gets too dark, the ball becomes very hard to see and so the game is stopped. \\
    \item Similarly, if it rains, the game is stopped due to the adverse affect this has on the pitch. \\
\end{itemize}
    
\end{frame}

\begin{frame}
    \frametitle{A motivating example}
    \begin{itemize}
        \item To illustrate the issue, consider the following example. \\
        \begin{example}
            Team A scores 320 runs in their 50 overs, losing 8 wickets in the process. While team B is batting, 
            it begins to rain, and the umpires call the game off with team B on 118-2 from 34 overs. After the rain stops,
            there is only time for 6 overs of play. 
        \end{example}
        \item Clearly, at this point it is unfair to expect team B to chase down 222 runs in 6 overs instead of the 16 they should 
            have had. So for this reason, score target adjustment is needed to keep the game fair, despite the loss of time.
\end{itemize}
        
\end{frame}

\begin{frame}
\frametitle{Duckworth, Lewis and Stern}

\begin{itemize}
    \item Statisticians Frank Duckworth and Tony Lewis set about a way to reduce score targets appropriately to overcome challenges
        like the one in the last example. \\
    \item They introduce the following formula
        \[
            Z(u,w) =  Z_0(w)(1e^{-b(w)u})
        .\] 
        Which gives the runs scored with $u$ overs remaining and $w$ wickets lost. Note that the actual value of $Z_0$ and $b$, the decay constant are not given due to commercial agreements.
\end{itemize}

\end{frame}

\begin{frame}
\frametitle{Problems with DLS}
foobar
    

\end{frame}

\section{Verse 2}
\begin{frame}
\frametitle{Esketit}
Oooh, lil pump
\end{frame}

\section{Conclusions}
\begin{frame}
\Huge{\centerline{Any Questions?}}
\end{frame}


\end{document} 
