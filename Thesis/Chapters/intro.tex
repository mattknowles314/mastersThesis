\chapter{Background}

\section{Limited-Overs Cricket and the need for Duckworth-Lewis}
The purest form of cricket, so-called ``first class" (FC) cricket, gave way to limited overs cricket in early 1950 in India. As attendance at \textit{County Championship} games in England began to drop, a limited overs tournamentm the ``Gillette Cup", was introduced in 1963. The idea that a game only lasting over the course of one day instead of four or five would boost crowd numbers. Originally, limited-overs and one day cricket were synonymous, but now limited-overs cricket comes in two forms: One Day International (50 Overs) and Twenty20 (20 Overs). Limited-overs cricket was designed to be played in knockout competition, which means a definitive result is attained on the day. This is in stark contrast to FC cricket, in which a draw is a highly likely result. \\


The first version of the Duckworth-Lewis-Stern method\footnote{Commonly still just reffered to as Duckworth-Lewis} (DLS) was published by Duckworth \cite{duckworth} in 1998, to solve a very specific problem. In limited overs cricket, when rain or bad-light interrupt play, and the full number of overs cannot be completed, DLS is used to revise the number of runs needed by the batting team in the second innings, to a total that would be considered an equivalent target. Duckworth states in the original paper that this method is designed in such a way that no team benefits or suffers from the shortening of the game. This is in contrast to unlimited-overs cricket, where a shortening of the game could actually massively favour the losing side into getting a draw. Due to the tournament-based nature of limited-overs cricket, a result is necessary, and so DLS is designed to reduce the likelehood of the two teams drawing. \\

\section{A Look at the Original Duckworth-Lewis Method}
This method was designed to fit 5 requirements. We summarise these requirements from the original paper in the following list.

\begin{enumerate}
	\item Method must be equally fair to both teams.
	\vspace{0.02cm}
	\item Results must be sensible in all possible scenarios.
	\vspace{0.02cm}
	\item Result should not depend on the scoring pattern of the first team.
	\vspace{0.02cm}
	\item Ease of application: requiring a table of numbers and a calculator.
	\vspace{0.02cm}
	\item Ease of understanding by all involved in cricket.
\end{enumerate}

Points (1-3) and (5) are perfectly understandable, however point (4) may be a little dated. With quick computational methods and access to computers, this requirement could be relaxed a little bit.

%Reproduce graph 2.1 in Python

\section{Stern's Extension}