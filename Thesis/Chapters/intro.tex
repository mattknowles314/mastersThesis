\chapter{Introduction}

\epigraph{I don't like discussing cricket off the field.}{MS Dhoni}

In this chapter, we outline the background behind this project. The discussion centres around why this a topic of relevancy in cricket, and how we can employ more modern mathematics to it. This comes after an introduction to cricket itself,
which the sports-inclined reader may still find of use, as it talks about the specific aspects of the game that will be most important in the remainder of the project. The issues raised by the current methods 
which we are trying to improve upon are explained. This leads to a review of some of the important literature on the topic to give more context behind the work done in this project.

\section{Data in Cricket}
The use of data in sport has become a massive part of how teams prepare for games and competitions in recent years. In cricket particularly, one way data has changed the game is the use
of metrics such as strike-rate for selecting bowlers in the starting XI. In previous eras, less substantive things such as cloud cover or how much grass covers the playing field would have taken precedance.
That's not to say such things don't still have a part to play, but the introduction of quantitiave methods has become much more commonplace in the modern game.
Similiarly, a coach may look at how wicket-taking in the ``powerplay''\footnote{In T20 cricket, this refers to the first 6 overs of the game, where certain fielding restrictions are in place}
vs in the ``middle overs'' will change the outcome of the game. Data visualisation is key in putting across this information to playing staff in order tp highlight key areas for
improvement. For this reason, we have taken an approach of using extensive data-visualisation to convey how the maths being done translates onto the cricket field. 

\section{Limited-Overs Cricket and the need for Duckworth-Lewis}
Cricket is a game played by two teams of eleven players. A coin toss decideds which team will ``bat'' first, and which will ``field'' fist. Unlike other sports, the scores in cricket are counted in ``runs''. Runs are achieved
in many ways, such as the two batters physically running to change sides of the pitch they are on, or by hitting the ball over a pre-defined boundary. The game of cricket itself has two main forms. In this project we will 
only look at ``limited-overs cricket''. Cricket is played in ``overs'', each one consisting of 6 individual bowls consecutively by the same bowler from one end of the pitch. Traditionally, there is no upper limit on the 
amount of overs a team may bowl- they keep going until the batting team ``declares'', meaning they've had enough and dont want to bat anymore, or preferably for the bowling team, are all bowled out. 
Batters can be given ``out'' in quite a few different ways, but once a batter is out- that's it for them. Once ten of the eleven players on the batting team have been given out, the innings comes to end and the team bowling 
gets their turn to bat. If they hit more runs before losing all their ten wickets or before the innings comes to an end, then they win. Cricket is an immensely complex game with countless intricacies and neuances, but at 
it's core, applying mathematics to it is not an absurd proposition. 

The traditional format of cricket, called ``first class'' lasts up to 4 days. This obviously has disadvantages from a spectator point of view. On
In 1963, the cricketing calender in the UK had a different format of the game ammended to it, alongside the usual first class fixtures.
The ``Gillette Cup'' introduced a form of cricket wherein each team only has 1 innings, lasting 50 overs isntead of the prior method of no limit. The idea was 
that this competition, coming to a conclusion within the space of a day, would increase spectator numbers and by extension,
ticket sales.\\

However, given the tournament-based nature of this new competition, we have the natural need for a definitive result.
This is something that is not always guarenteed in first class cricket, where draws are common.\footnote{Note that 
``draw'' and ``tie'' are not interchangable terminology in cricketing terms.} As such, in order to allow for a result to be
determined when a game is cut short, the idea of target-revision was introduced. This meant that a team could be set a
roughly equivalent run target off of a reducded number of overs that would still classify them as the winners.
This is where Duckworth and Lewis came in, proposing their D/L method \cite{duckworth}. As will be discussed in Chapter 3, the D/L method 
has been slighlty improved upon since it was published in the late 1990s, but the overarching philosophy of the method has stayed the same. 
One extension was that of Stern in 2004, \cite{stern}, now  known as the DLS method. This allowed for better score resetting in Twenty-20 (T20) cricket- an even shorter 
form of the game introduced to last only over the course of an afternoon instead of a day. 

\section{Problems raised with DLS}
In $\cite{phanse}$, the authors found that DLS has a bias towards not only the team batting first, but whoever won the coin toss at the start of the game.
The winner of the toss chooses whether their side will bat or bowl in the first innings. They go onto to propose a simple extension to DLS for reducing
these biases, but we won't cover that in this project. 

\section{Project Aims}
The main aim of this project is to look at methods for \textit{predicting} cricket scores. It is important to make the distinction between this 
and \textit{projecting} the scores, which is what goes on in the game at the minute. Projecting scores assumes a constant runrate, which it will be seen 
in later chapters is not really the case. However, we can make use of the patterns in run rates to try and extract a predictable score. 
%should probably try and phrase this a little bit better?


\section{Review of current Literature}
We begin the look at literature on this topic with the paper published by F. Duckworth himself in 1998 $\cite{duckworth}$. In this paper, 
Duckworth proposes the`D/L method based on 5 principles. However, there is an issue with the sentence that appears after defining the function $Z(u,w)$.
``Comercial confidentiality prevents the disclosure of the mathematical definitions of these functions'' This means we can't access the actual parameters. 
that they use for defining the model. This won't affect the development of our model, but it is unfortunate as it means getting a clear picture of DLS is slightly harder.
The claim is that these functions have been defined via experimentation. This however gives rise to the first issue: the first T20 game of cricket was not played until 2003. 
So clearly, D/L was not designed with T20 in mind, and so the functions derived from experementation and research will not be accurate for T20 cricket. \\

In 2004, a year after the first T20 matches were played, Duckworth and Lewis published another paper $\cite{duckworth2}$, in which they report
that the table used for calculating the method can be employed by anyone at all levels of the game- from grass-roots cricket up to the highest standard of 
the professional game. 

Another paper that is worth noting is by Saqlain et. al. \cite{saqlain}, looking at predicting scores from the 2019 Cricket World Cup, an ODI
tournament held across England. They used data after the previous world cup, held in Australia and New Zealand in 2015; all the way to $10^{th}$ November
2018. The interesting thing about this paper, is they do not use data from individual matches, but instead on 10 meta-statics, including 
``Number of series victories'', ``total wickets'', ``centuries'' and ``all out'', among others. They apply the TOPSIS method to this specific problem, using the 
following algorithm. \\
\begin{algorithm}
\caption{Modified TOPSIS Algorithm}\label{TOPSIS}
\begin{algorithmic}[1]
    \Require $m,n \geq 0$
    \State $\text{Calculate normalised matrix} \ X_{ij} = \frac{x_{ij}}{\sqrt{\sum_{i=1}^mx_{ij}^2}}$
    \State $\text{Calculate weighted normalised matrix} \ V_{ij}=X_{ij}W_j$
    \State $\text{Calculate Euclidean distance from the best outcome } \ S_i^+ = \sqrt{\sum_{j=1}^n(V_{ij}-V_j^+)^2} \ \text{for} \ i \in \{1,2,...,m\}$
    \State $\text{Calculate Euclidean distance from the worst outcome } \ S_i^- = \sqrt{\sum_{j=1}^n(V_{ij}-V_j^-)^2} \ \text{for} \ i \in \{1,2,...,m\}$
    \State $\text{Calculate performance score} \ C_i^* = \frac{S_i^-}{S_i^+  + S_i^-}$
\end{algorithmic}
\end{algorithm}

They used the results of this process to predict how the 2019 World Cup would look, rather than games for individual scores. However, they were largely unsuccessful. 
Not only did the authors leave Sri Lanka out of their calculations entirely, they only succesffuly predicted India would finish top of the points table, and Afghanistan 
at the bottom. It's worth mentioning that this may not be an issue with the mathematical model iteself, but the imbalance of data. Teams like Afghanistan play far less games 
than a team like England or Inida. This introduces a bias favouring teams who have played more games not only from a statistical point of view, but from a cricketing one too 
since the teams that play less will have less match practice going into a big tournament like the World Cup. \\

However, a paper by Kumar and Roy, \cite{kumar} takes an approach along the same lines of the aim of this project. It is not necessary to discuss the individual aspects
of their paper here, as it will be discussed when we dive into the methods themselves. However it is worth noting the results of this paper now to see how
ours differ. They found their limited dataset to be a problem in classifying. This is something we have been aware of, but there isn't much that can be done
given the nature of how mmany cricket games are played year on year. 