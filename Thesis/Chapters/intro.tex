\chapter{Background}

\section{Limited-Overs Cricket and the need for Duckworth-Lewis}
Cricket is a game played by two teams, each with 11 players. The objective for the batting team is to score as many 'runs' as possible
without losing 10 of their batsmen\footnote{There have to be two batters on the pitch, so if 10 wickets are lost, it leaves one batter stranded.}, 
who can be 'out' in a variety of ways. Cricket is played in 'overs', each over lasting 6 deliveries. Traditionally, the game lasted for 4 days
and there was no limit to the number of overs the bowling team could bowl. 

In 1963, the cricketing calender in the UK had for the first time a different format of the game ammended to it.
The ``Gillette Cup'' introduced a form of cricket wherein each team has 1 innings, lasting 50 overs. The idea was 
that this competition, coming to a conclusion within the space of a day, would increase spectator numbers and by extension,
ticket sales.

However, given the tournament-based nature of this new competition, we have the natural need for a definitive result.
This is something that is not always guarenteed in "first class" cricket, where draws are common. \footnote{Note that 
"draw" and "tie" are not interchangable terminology in cricketing terms}. As such, in order to allow for a result to be
determined when a game is cut short, the idea of target-revision was introduced. This meant that a team could be set a
roughly equivalent run target off of a reducded number of overs that would still classify them as the winners.

\section{Problems raised with DLS}
foo

\section{Project Aims}
bar

\section{Review of current Literature}
We begin the look at literature on this topic with the paper published by F. Duckworth himself in 1998 \cite{duckworth}. In this paper, 
Duckworth proposes the ``D/L'' method based on 5 principles. HOweverm the issue with the sentence that appears after defining the function $Z(u,w)$.
``Comercial confidentiality prevents the disclosure of the mathematical definitions of these functions''. The claim is that these functions have been 
via experimentation. This however gives rise to the first issue: the first T20 game of cricket was not played until 2003. So clearly, D/L was not
designed with T20 in mind, and so the functions derived from experementation and research will not be accurate for T20 cricket. 

In 2004, a year after the first T20 matches were played, Duckworth and Lewis published another paper \cite{duckworth2}, in which they report
that the table used for calculating the method can be employed by 
