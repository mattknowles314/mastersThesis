\chapter{Introduction}

\section{Data in Cricket}
The use of data in sport has become a massive part of how teams prepare for games and competitions in recent years. In cricket particularly, one thing is the use
of metrics such as strike-rate for selecting bowlers for the starting XI. In limited overs cricket, asking questions about how scoring boundaries will affect
the score at different grounds. Or how wicket-taking in the ``powerplay''\footnote{In T20 cricket, this refers to the first 6 overs of the game, where certain fielding restrictions are in place}
vs in the ``middle overs'' will change the outcome of the game. Data visualisation is key in putting across this information to playing staff to highlight key areas for
imprtovement. 

\section{Limited-Overs Cricket and the need for Duckworth-Lewis}
Cricket is a game played by two teams, each with 11 players. The objective for the batting team is to score as many ``runs'' as possible
without losing 10 of their batsmen\footnote{There have to be two batters on the pitch, so if 10 wickets are lost, it leaves one batter stranded.}, 
who can be ``out'' in a variety of ways. Cricket is played in ``overs'', each over lasting 6 deliveries. Traditionally, the game lasted for 4 days
and there was no limit to the number of overs the bowling team could bowl. \\

In 1963, the cricketing calender in the UK had for the first time a different format of the game ammended to it.
The ``Gillette Cup'' introduced a form of cricket wherein each team has 1 innings, lasting 50 overs. The idea was 
that this competition, coming to a conclusion within the space of a day, would increase spectator numbers and by extension,
ticket sales.\\

However, given the tournament-based nature of this new competition, we have the natural need for a definitive result.
This is something that is not always guarenteed in "first class" cricket, where draws are common. \footnote{Note that 
``draw'' and ``tie'' are not interchangable terminology in cricketing terms.}. As such, in order to allow for a result to be
determined when a game is cut short, the idea of target-revision was introduced. This meant that a team could be set a
roughly equivalent run target off of a reducded number of overs that would still classify them as the winners.

\section{Problems raised with DLS}
In $\cite{phanse}$, the authors found that DLS has a bias towards not only the team batting first, but whoever won the coin toss at the start of the game.
The winner of the toss chooses whether their side will bat or bowl in the first innings. They go onto to propose a simple extension for reducing
these biases, but we won't cover that in this project. 

\section{Project Aims}
The main aim of this project is to look at methods for \textit{predicting} cricket scores. It is important to make the distinction between this 
and \textit{projecting} the scores, which is what goes on in the game at the minute. Projecting scores assumes a constant runrate, which it will be seen 
in later chapters is not really the case. However, we can make use of the patterns in run rates to try and extract a predictable score. 
%should probably try and phrase this a little bit better?


\section{Review of current Literature}
We begin the look at literature on this topic with the paper published by F. Duckworth himself in 1998 $\cite{duckworth}$. In this paper, 
Duckworth proposes the ``D/L'' method based on 5 principles. Hwever, the issue with the sentence that appears after defining the function $Z(u,w)$.
``Comercial confidentiality prevents the disclosure of the mathematical definitions of these functions''. The claim is that these functions have been 
defined via experimentation. This however gives rise to the first issue: the first T20 game of cricket was not played until 2003. So clearly, D/L was not
designed with T20 in mind, and so the functions derived from experementation and research will not be accurate for T20 cricket. \\

In 2004, a year after the first T20 matches were played, Duckworth and Lewis published another paper $\cite{duckworth2}$, in which they report
that the table used for calculating the method can be employed by 

Another paper that is worth noting is by Saqlain et. al. \cite{saqlain}, looking at predicting scores from the 2019 Cricket World Cup, an ODI
tournament held across England. They used data after the previous world cup, held in Australia and New Zealand in 2015; all the way to $10^{th}$ November
2018. The interesting thing about this paper, is they do not use data from individual matches, but instead on 10 meta-statics, including 
``Number of series victories'', ``total wickets'', ``centuries'' and ``all out'', among others. They apply the TOPSIS method to this specific problem, using the 
following algorithm. \\

\begin{algorithmic}[1]
    \State $\text{Calculate normalised matrix} \ X_{ij} = \frac{x_{ij}}{\sqrt{\sum_{i=1}^mx_{ij}^2}}$
    \State $\text{Calculate weighted normalised matrix} \ V_{ij}=X_{ij}W_j$
    \State $\text{Calculate Euclidean distance from the best outcome } \ S_i^+ = \sqrt{\sum_{j=1}^n(V_{ij}-V_j^+)^2} \ \text{for} \ i \in \{1,2,...,m\}$
    \State $\text{Calculate Euclidean distance from the worst outcome } \ S_i^- = \sqrt{\sum_{j=1}^n(V_{ij}-V_j^-)^2} \ \text{for} \ i \in \{1,2,...,m\}$
    \State $\text{Calculate performance score} \ C_i^* = \frac{S_i^-}{S_i^+  + S_i^-}$
\end{algorithmic}

They used the results of this process to predict how the 2019 world cup would look, rather than games for individual scores. However, they were largely unsuccessful. 
Not only did the authors leave Sri Lanka out of their calculations entirely, they only succesffuly predicted India would finish top of the points table, and Afghanistan 
at the bottom. 

However, a paper by Kumar and Roy, \cite{kumar} takes an approach along the same lines of the aim of this project. It is not necessary to discuss the individual aspects
of their paper here, as it will be discussed when we dive into the methods themselves. However it is worth noting the results of this paper now to see how
ours differ. They found their limited dataset to be a problem in classifying. This is something we have been aware of, but there isn't much that can be done
given the nature of how mmany cricket games are played year on year. 