\chapter{The CricNet Package}

This chapter of the appendix contains no new mathematics or results. Instead, we discuss the process of tidying up all the code used throughout this project into an R package that can be used and extended for future work on this topic. By publishing this code as an R package, it allows the curious reader to explore the things we have discussed in previous chapters with ease, rather than having to install all necessary packages and data.\\

\section{Introduction}

R as a language is built on packages, the \textbf{C}omprehensive \textbf{R} \textbf{A}rchive \textbf{N}etwork, (CRAN) network stores 
thousands of packages that can be used for performing tasks in R, to save re-inventing the wheel. In this project, we have used CRAN to obtain 
access to the \verb|neuralnet| package and others. Packages are installed from CRAN using the function \textit{install.packages("[package]")} and loaded with \textit{library("[package]"}. However, due to the relatively small scale of this project, we won't be storing the package on CRAN (at this point in time), but rather on Github. This is preferable due to how easy it is to access code on Github, and because it also allows other people to directly contribute to the expansion of the package should they wish.  

\section{Building an R Package}

Building an R package is quite a simple concept- we are just putting all our data and functions into one place that can allow for easy modification and the replicating of results. Actually building a package requires using other packages, specifically the \verb|devtools| package. Running the function \textit{usethis::create_package()} from the \verb|usethis| package. This function creates a directory with the necessary files for an R package. This file structire initially can be seen in the directory tree

\dirtree{%
.1 CricNet.
.2 CricNet.Rproj.
.2 DESCRIPTION.
.2 NAMESPACE.
.2 R.
.2 data.
}

In the R directory is unsurprisingly where the R files containing all the functions go. Naturally not every piece of code we wrote will go into the package, as some of it was simply for demonstrative purposes. All the code for actually running the network (which is more or less a wrapper on the \verb|neuralnet| package), along with the code for analysing those results goes into the package.
Each function gets its own file, for ease of access and debugging. 

\section{CricNet Structure}

The documentation for what each function does can be found within the package documentation, or using \textit{help([function])} in the R consone.The purpose of this section is to give an overview of the package and how to use to replicate results from this project.
The CSV containg the runrates the network was initially trained on exits in the \textit{data} directory as ``rrmat.rda''. Saving this data is done using the function \textit{usethis::use_data()} function which puts everything in the right format and direcotry automatically. There are four R functions that come with the package, ``scoreNet.R'', ``genResults.R'', ``unscale'' and ``netAnalysis.R''. Note that they should be used in this order. No output comes from the first two, but by running them, several R objects are created. One for the network, and one for storing the results. The analysis script takes this results dataframe and displays a corrolation score and a a density plot of the errors. In the future, more analysis features will be added to this function. For now, this is inline with the work that has been done in this project. \\

Adding documentation to each function is done using the \verb|roxygen2| package. It is a case of simply writing docstrings above the functions in their respective files, and using \textit{usethis::document()} to build the file. These pieces of documentation create new ``.Rd'' files in the ``/man'' directory. Storing them here allows them to be viewed alone by running the command \textit{help([function])} or \textit{?[function]} in an R terminal window. 

The overall structure of the \verb|CricNet| package can be seen in the directory tree.

\dirtree{%
.1 CricNet.
.2 CricNet.Rproj.
.2 data.
.3 rrmat.rda.
.2 DESCRIPTION.
.2 man.
.3 genResults.Rd.
.3 netAnalysis.Rd.
.3 scoreNet.Rd.
.3 unscale.Rd.
.2 NAMESPACE.
.2 R.
.3 genResults.R.
.3 netAnalysis.R.
.3 scoreNet.R.
.3 unscale.R.
.2 readme.md
}

\section{Using \verb|cricnet|}
The package can be accessed at ``https://github.com/mattknowles314/CricNet''. It can be installed directly in R using the \verb|devtools| function \textit{install_github(mattknowles314/CricNet)}. Any bugs or issues can be reported via the ``Issues'' tab on the Github page. 
