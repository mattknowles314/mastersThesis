\chapter{The CricNet Package}

This chapter contains no new mathematics or results. We discuss turning the code written here into an R package that can be used for easily expanding on the work of this project.
The package comes complete with the dataset that we trained our network one, meaning the interested reader could replicate our results or even improve upon them. 

\section{Introduction}

R as a language is built on packages, the \textbf{C}omprehensive \textbf{R} \textbf{A}rchive \textbf{N}etwork, CRAN, stores thousands of packages that can be used for performing tasks in R. This saves people re-inventing the wheel when performing statistical analyses. In this project, we have used CRAN to obtain 
access to the several packages that were imperitive in undertaking the task of building a neural network. 
Packages are installed from CRAN using the function \textit{install.packages(``package'')} and loaded with \textit{library(``package'')}. However, due to the relatively small scale of this project, we won't be storing the package on CRAN (at this point in time), but rather on Github. This is preferable due to how easy it is to access code on Github, and because it also allows other people to directly contribute to the expansion of the package should they wish.  

\section{Building an R Package}

Building an R package is quite a simple concept- we are just putting all our data and functions into one place that can allow for easy modification and the replicating of results. Actually building a package requires using other packages, specifically the \verb|devtools| package. Running the function \textit{usethis::create\_package()} from the \verb|usethis| package creates a directory with the necessary files for an R package. This file structire initially can be seen in the directory tree below. \\

\begin{figure}[h]
\caption{Default File Structure for an R Package}

    \dirtree{%
    .1 CricNet.
    .2 CricNet.Rproj.
    .2 DESCRIPTION.
    .2 NAMESPACE.
    .2 R.
    .2 data.
    }    



\end{figure}

The R directory is unsurprisingly where the R files containing all the functions go. Naturally not every piece of code we wrote will go into the package, as some of it was simply for demonstrative purposes. All the code for actually running the network (which is more or less a wrapper on the \verb|neuralnet| package), along with the code for analysing those results goes into the package.
Each function gets its own file, for ease of access, documentation and debugging. \\

The files NAMESPACE and DESCRIPTION are metadata files used by the \verb|roxygen2| package for mainting things such as version number, liscencing and name(s) of the author(s). NAMESPACE is also used to store any dependencies on packages that the package relies on. For example, our \textit{scoreNet.R} file, as mentioned, is a wrapper on the function \textit{neuralnet::nerualnet()}, so it clearly needs to import the required package.


\section{CricNet Structure}

The documentation for what each function does can be found within the package documentation, or using \textit{help(function)} in the R console. The purpose of this section is to give an overview of the package and how to use to replicate results from this project. \\

The CSV containg the run rates that the network was initially trained on exits in the \textit{data} directory as ``rrmat.rda''. Saving this data is done using the function \textit{usethis::use\_data()} function which puts everything in the right format and direcotry automatically. There are four R functions that come with the package, ``scoreNet.R'', ``genResults.R'', ``unscale.R'' and ``netAnalysis.R''. Note that they should be used in this order. No output comes from the first two, but by running them, several R objects are created. One for the network itself (an object of type \textit{nn}), and one for storing the results. The analysis script takes this results dataframe and displays a corrolation score and a density plot of the errors. In the future, more analysis features will be added to this function. Release version 0.1 of the package is the one that corresponds directly to the work of this project, with no expansions.\\

Adding documentation to each function is done using the \verb|roxygen2| package. It is a case of simply writing docstrings above the functions in their respective files, and using \textit{usethis::document()} to build the file. In the docstring, the ``export'' tag is what allows the functions to be used directly by anyone who installs the package.  
These pieces of documentation create new ``.Rd'' files in the ``/man'' directory. Storing them here allows them to be viewed alone by running the command \textit{help(``function'')} in an R console. 

The overall structure of the \verb|CricNet| package can be seen in the directory tree in Figure~\ref{cricnetstruct}.\\

\begin{figure}[ht]
\caption{File structure for the CricNet package}

    
\dirtree{%
.1 CricNet.
.2 CricNet.Rproj.
.2 data.
.3 rrmat.rda.
.2 DESCRIPTION.
.2 man.
.3 genResults.Rd.
.3 netAnalysis.Rd.
.3 scoreNet.Rd.
.3 unscale.Rd.
.2 NAMESPACE.
.2 R.
.3 genResults.R.
.3 netAnalysis.R.
.3 scoreNet.R.
.3 unscale.R.
.2 readme.md.
}

\label{cricnetstruct}
\end{figure}

\section{Using The Package}
The package can be accessed at ``https://github.com/mattknowles314/CricNet''. It can be installed directly in R using the \verb|devtools| function \textit{install\_github(mattknowles314/CricNet)}. Any bugs or issues can be reported via the ``Issues'' tab on the Github page. 
