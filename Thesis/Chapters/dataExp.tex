\chapter{The Data and Associated Methods}

\section{Data Wrangling Methodology}
The data that forms the backbone of this project is courtesy of \cite{cricData}, which provides ball-by-ball match data for over 2000 One-Day International matches
and over 1500 Twenty20 matches. The ball-by-ball aspect of this data isn't immediately very useful, but the extra match information will be used extensively throught
this paper. Each match comes in the form of a .JSON file, so must be decoded first in order to make use of. All the code that will be discussed throught this section
can be found in appendix one. \newline



In order to get the data into a usable form, we wrote several Python programs to comb through and select datapoints that were of interest. Firstly, the data was split 
into games that were decided on DLS, and those that played out a full innings. This resulted in a CSV file containing information on 190 ODIs that were decided on DLS. 
The datapoints collected were the home and away team, the ground, the winner of the game, how much they won by, and what their target was. 
The rationale behind adding the ground is that the ground a game is played at has a considerable influence on how high a score will go. For example, 
consider the two grounds Edgbaston and Trent Bridge, in Birmingham and Nottingham respectively. The follwing table shows the averages of 6 international teams at the two 
English grounds in their entire ODI playing history.

%https://stats.espncricinfo.com/ci/engine/ground/56788.html?class=2;orderby=runs;template=results;type=aggregate
%https://stats.espncricinfo.com/ci/engine/ground/57219.html?class=2;orderby=runs;template=results;type=aggregate

\begin{center}
    \begin{tabular}{c|c|c}
    Team & Edgbaston Average & Trent Brdige Average \\
    \hline
    Australia & 196.3 & 272.9 \\  
    Bangladesh & 211.5 & 268.7 \\
    England & 227.8 & 246.1 \\
    India & 224.7 & 226.3 \\
    New Zealand & 202.2 & 241.9 \\
    Pakistan & 189.6 & 250.4 \\
    Overall & 208.7 & 251.1    
    \end{tabular}
\end{center}

So one can see that on average, a team playing at Trent Bridge is going to have a high scoring game than a team playing at Edgebaston. But this is something that DLS does
not take into account when revising the score. It is for this reason the ground has been included in the data.