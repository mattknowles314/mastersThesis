\documentclass[11pt]{report}
\usepackage[utf8]{inputenc}
\usepackage[a4paper, total={6in,8in}, portrait, margin=1in]{geometry}
\usepackage{amssymb}
\usepackage{array}
\usepackage{graphicx}
\usepackage{amsmath}
\usepackage{amsfonts}
\usepackage{float}
\usepackage{mathtools}
\usepackage{listings}
\usepackage{xcolor}
\usepackage[misc]{ifsym}
\usepackage{indentfirst} 
\usepackage{amsthm}
\usepackage{appendix}
\usepackage{caption}
\usepackage{subcaption}
\usepackage{fancyhdr}
\usepackage{algorithm}
\usepackage{algpseudocode}
\usepackage{tikz}
\usepackage{dirtree}
\usepackage{epigraph}
\usetikzlibrary{positioning}
\pagestyle{fancy}


%Headers and Footers, thanks to Overleaf
%https://www.overleaf.com/learn/latex/How_to_Write_a_Thesis_in_LaTeX_(Part_2):_Page_Layout
\fancyhead{}
\fancyhead[RO,LE]{\small Improving Duckworth-Lewis: Statistical Methods for Revising Score Targets in Limited-Overs Cricket}
\fancyfoot{}
\fancyfoot[LE,RO]{\thepage}
\fancyfoot[LO,CE]{Chapter \thechapter}
\fancyfoot[CO,RE]{Matthew Knowles}

\DeclarePairedDelimiter{\ceil}{\lceil}{\rceil}

\newtheorem{theorem}{Theorem}[section]
\newtheorem{remark}[theorem]{Remark}
\newtheorem{definition}[theorem]{Definition}
\newtheorem{example}[theorem]{Example}
\newtheorem{lemma}[theorem]{Lemma}

\definecolor{codegreen}{rgb}{0,0.6,0}
\definecolor{codegray}{rgb}{0.5,0.5,0.5}
\definecolor{codepurple}{rgb}{0.58,0,0.82}
\definecolor{backcolour}{rgb}{0.95,0.95,0.92}

\lstdefinestyle{mystyle}{
    backgroundcolor=\color{backcolour},   
    commentstyle=\color{codegreen},
    keywordstyle=\color{magenta},
    numberstyle=\tiny\color{codegray},
    stringstyle=\color{codepurple},
    basicstyle=\ttfamily\footnotesize,
    breakatwhitespace=false,         
    breaklines=true,                 
    captionpos=b,                    
    keepspaces=true,                 
    numbers=left,                    
    numbersep=5pt,                  
    showspaces=false,                
    showstringspaces=false,
    showtabs=false,                  
    tabsize=2
}

\lstset{style=mystyle}

\begin{document}

\begin{titlepage}

    \begin{center}

        \vspace*{1.5cm}

        \textbf{\huge Improving Duckworth-Lewis: Statistical Methods for Revising Score Targets in Limited-Overs Cricket} 
        
        \vspace*{1.5cm}
        
        \textit{Matthew Knowles}

        \vspace*{1.5cm}
    
        \includegraphics[scale=0.5]{figures/uoylogo.png}
        
        \vspace*{1.5cm}
        
        Department of Mathematics, \\
        University of York, \\
        United Kingdom \\

        \vspace*{1.5cm}

        \textit{A dissertation submitted in partial fulfillment of the requirements for the degree of Master of Mathematics}
    \end{center}
    
\end{titlepage}

\section*{Abstract}
A Neural Network model for predicting scores in the sport of Cricket is presented. The aim is to use our model to improve upon the results of the method currently employed by 
the International Cricket Council (ICC). We explore this current method, known as the DLS method \cite{stern} in 
some detail before going on to develop our own model. This is followed by some exploratory data analysis, which is imperative in making certain assumptions that aid 
in the building of the Neural Network. For the Neural Network, we take an in-depth look at the mathematical underpinning of building and training such a model. This 
is accompanied by code snippets demonstrating how these methods are implemented in a high-level programming language. We then perform an analysis of the results 
of our model before applying it to the data from the ICC 2019 Cricket World Cup, in order to see how it faired against the DLS. 
We find that our method does not predict scores perfectly, but does produce slightly more attainable score targets than DLS- as illustrated by the World Cup simulation 
study. 

\section*{Acknowledgements}
The author would like to thank Dr. Jessica Hargreaves for their invaluable supervision, encouragement and advice throughout this project. In addition,
many thanks to Michael Najdan at Kent County Cricket Club for their insight into how data is used in the professional game of cricket, and to Harrison Allen at Yorkshire County Cricket Club for demonstrating how limited-overs data was put into practice in preperation for games. 
I couldn't write a dissertation about cricket without mentioning the University of York Men's Cricket Club- it has been wonderful to share the playing field with you all over the past 4 years.
Finally, I would like to dedicate this dissertation to my friends from Lindley G; thank you for being the most wonderful and supportive people I've ever met. You'll never know how much you mean to me.

\section*{Statment of Originallity}
The enclosed content is my own work. Any work taken from another source has been appropriately acknowledged and referenced. 

\section*{Notes}
All code written in support of this project can be found on GitHub at: \\
\begin{itemize}
    \item \textit{https://github.com/mattknowles314/mmathThesis}. 
    \item \textit{https://github.com/mattknowles314/CricNet}.
\end{itemize}


\setcounter{tocdepth}{3}
\tableofcontents
\setcounter{tocdepth}{1}
\listoffigures


\input{Chapters/Intro}

\chapter{Data}

\section{Source}
foo

\section{Attributes}
bar

\section{Pre-Processing}
foobar


\chapter{The DLS Method in Detail}

We now look at the mathematics behind the D/L, and DLS methods. D/L being the original
method, and DLS the method that Stern helped to revise. In the original paper, the authors
state ``Commercial confidentiality prevents the disclosure of the mathematical definitions
of these functions.'' $\cite{duckworth}$. Which is naturally a slight problem for this, but what
we can do is instead use sample values based on data we have to look at how these functions behave,
so all is certainly not lost.

\section{Origins: Duckworth and Lewis}
We begin by looking at the original paper. The first thing to establish is how many runs are scored,
on average, in a given number of overs. This is given by the equation:

\begin{equation}
    Z(u) = Z_0[1-exp(-bu)]
    \label{Z(u)}  
\end{equation}

Where u is the number of overs, b is the exponential decay constant, and $Z_0$ is the
average total score in first class cricket, but with one-day rules imposed.  

Now because we don't have access to actual values for $Z_0$ or b, a plot to see what equation \ref{Z(u)} looks 
like was created by using 3 sample values for $Z_0$. For b, it was a process of trial and error to find a value
that resulted in the graph having a similar shape to original figures in the D/L paper. 

\begin{figure}[h]
    \centering
    \includegraphics[scale=0.6]{figures/z(u).png}
    \label{Z(u)_graph}
    \caption{Graph showing how the rate at which runs are scored decays as a game progresses.}
\end{figure}

However, we have not yet looked at what happens when wickets are lost. To introduce this metric, equation \ref{Z(u)}
is revised to incorperate the scenario that w wickets have been lost, and that there are u overs remaining
The revised equation is given as follows:

\begin{equation}
    Z(u,w) = Z_0(w)[1-exp(-b(w)u)]
    \label{zuw}
\end{equation}

Where now, we have $Z_0(w)$ giving the average total score from the last $10-w$ wickets in first class cricket.
We now also have $b(w)$ as the exponential decay constant, which now changes depending on wickets lost.

%Should probably make a plot of this, to go with the other one, but due to confientiality
%it's quite hard to get the numbers for Z0(u,w) myself.

With this in mind, we now look at the specific case of equation \ref{zuw} with $u=N$ and $w=0$, namely, the conditions
at the start of an N-over innings. We have:

\begin{equation}
    Z(N,0) = Z_0[1-exp(-bN)].
    \label{zstart}
\end{equation}

Which we then incorperate into the ratio

\begin{equation}
    P(u,w) = \frac{Z(u,w)}{Z(N,0)}.
    \label{prat}
\end{equation}

The ratio \ref{prat} gives, keeping in mind there are u overs still to be bowled, with w wickets
lost, the average proportion of the runs that still need to be scored in the innings. 
It is this ratio that is where the revised scores come from. Let us now look a bit more at how that works
practically.

\begin{example}
    \label{dlExMain}
    Assume there is a break in the second innings (due to rain or similar), which results in the second team missing some overs.
    Let $u_1, u_2$ be the number of overs played before the break, and available after it respectively. We impose the condition
    that $u_2 < u_1$. At the time of the break, the second team had lost $w$ wickets. The aim is to adjst the required score to account
    for the $u_1 - u_2$ overs they have lost. The winning ``resources'' available are given by

    \[
        R_2 = [1-P(u_1,w)+P(u_2,w)].
    \]  

    Which means, if the first team batting scored S runs, then the new tartget is given by

    \[
        T = \ceil{SR_2}
    \]  
\end{example}

\section{Improvements by Stern}
foo


\section{Extension: Bayesian Modelling}
This section will look at the work of paper $\cite{dlbayes}$, which used Bayesian modelling to try and improve the score predictions
of the D/L method. The authors used the same dataset as we are using for the dissertation, although over a slightly smaller range of games.
Only games between 2005-2017 are used, whereas our dataset goes up to 2021. Note that to keep consistent with the fact we are talking about a different
model now, we swich from using $Z(u,w)$, to using $R(u,w)$, as to keep consistency with the different papers being looked at.
They start by introducing the following nonlinear regression model:

\begin{equation}
    \label{dlRegress}
    \bar{R}(u,w) \sim N(m(u,w;\theta),\frac{\sigma^2}{n_{uw}})
\end{equation}

Where $\bar{R}(u,w)$ is the sample average of runs scored by a team from the total number of matches in the data set. We have $m(u,w;\theta)$
as the corresponding modeled population average of runs scored by a team when a considerable amount of games are taken into account. $\theta$ denotes 
a vector of parameters that will be specified later on. Since $R(u,w)$ is not calculated in all games, the average score is taken over all matches where
scores are present, this is given by $n_{uw}$. The sample mean $\bar{R}(u,w)$ is the calculated over this quantity. This is actually the point which motivates
using Bayesian statistics to extend the D/L method. The authors report that approximately $26.8\%$ of values for $R(u,w)$ were missing in the dataset, so by 
using a Bayesian inference model, we can use the posterior predictive distribution to account for missing values. Which in turn should increase the 
predictive accuracy of this model.

We return now to look at the $m(u,w;\theta)$ parameter that appears in \ref{dlRegress}. This mean function, based on the plots produced in the original D/L
paper, (see \ref{Z(u)_graph} for an example), the authors adopted an exponential decay. We can see why this choice makes sense from figure \ref{Z(u)_graph}.
The resources considered by this method, namely runs and wickets, do decrease exponentially as a game progresses. $m(u,w;\theta)$ depends on the parameters
$a_w = Z_0(w)$ and $b_w = b(w)$, each one depending on the number of wickets fallen at that given time, with u overs remaining. We therefore have:

\begin{align}
    \label{mdef}
        m(u,w;\theta) &= a_w(1-exp(-b_wu)) \\
               \theta &= \{ (a_w,b_w); w \in \mathcal{W}=\{0,1,\ldots,9\} \} 
\end{align}

Before proceeding with defining the prior specifications on the parameters for this model, we need the following distribution:

\begin{definition}
    $X \sim Ga(a,b)$ has a \textbf{Gamma Distribution} with mean $\frac{a}{b}$ and variance $\frac{a}{b^2}$ if its probability density function is
    $$
        \frac{b^a}{\Gamma(a)} x^{a-1}e^{-bx}.  
    $$
\end{definition}

We can now define the prior specifications on the parameters. First, we fix $A_0$ and $B_0$ large enough such that $R(50,0) < A_0$. We initialise  the model with
$a_0 \sim U(0,A_0)$ and $b_0 \sim U(0,B_0)$. Then given any pair $(a_0,b_0)$ and for $w=0,1,\ldots,8$, we generate:

\begin{align}
    a_{w+1}|\sigma^2,a_w,b_w &\sim U(0,a_w) \\
    b_{w+1}|\sigma^2,a_{w+1},b_w,a_w &\sim U\left(0,\frac{a_wb_w}{a_{w+1}}\right) \\
    \frac{1}{\sigma^2} &\sim Ga(a,b).
\end{align}

Above, U(a,b) represents a uniform distribution over the open interval $(a,b)$. Now that the prior distribution is obtained, the likelehood function is then 
given by:

\begin{equation}
    L(\theta,\sigma^2) = \left( \frac{1}{\sigma / \sqrt{n_{uw}}} \right)^{500} exp \left\{ - \frac{1}{2(\sigma^2 / n_{uw})} \sum_{u=1}^{50} \sum_{w=0}^9 (R(u,w)-m(u,a_w,b_w))^2  \right\}
\end{equation}

The authors chose their parameters $A_0, \ B_0, \ a \ \text{and} \ b$ such that the prior is not sensitive to the posterior inderence. The values chosen
were therefore $A_0=200 \ B_0=100 \ \text{and} \ a=b=0.1$.

One thing briefly mentioned in chapter 3 of this paper is the average score at the fall of each wicket. This isn't discussed further, but it's interesting to see how it behaves.
So we take a brief detour to look at it. This can be seen in figure \ref{avgrunsfow}.

\newpage

\begin{figure}[t]
    \centering
    \includegraphics[scale=0.6]{figures/avgrunsfow.png}
    \caption{Plot of average runs scored at the fall of each wicket. Averages taken over 1437 innings.}
    \label{avgrunsfow}
\end{figure}

Initially, one may think that this is normally distributed. It certainly holds a bell curve-like shape. However, a Kolmogorov-Smirnov test $\cite{kolm}$ was
performed to see if these values were normally distributed, and it turns out they aren't. This test was performed using the R function \textit{ks.test()}, which returned
a p-value of $2.2\text{x}10^{-16}$ for the two-sided alternative hypothesis parameter. 

Does the shape of this curve make sense? Well yes, because usually batters 3/4 are actually the best in the side, so the fact they put on more runs is not a suprise.
But what this does do is reinforce the idea behind using an exponential decay for modelling resources. This is because cumulatively, the tail end of batters (the last 4) 
will not put on as many runs as the higher/middle order. 

Most of the work from this paper went into creating a better resource table for calculating revised scores. Which isn't particularly relevant to this project, but what is 
is the section on score prediction. The authors split every match at the $35^{th}$ over to try and predict the final score. They found the bayesian model gives a better
prediction of final match scores. Figure 3 of their paper outlines these results.

\chapter{Exploratory Data Analysis}

In this chapter, we are looking only at the first innings of the games, and only those games in which the full 50 overs were played. The 
reason for this is the models we will build are going to try and predict a score as if a full innings has been played. \\

We begin our exploration of the data with a look at how the density of the runs scored per fall of wicket changes. This has been done for each
individual team in the dataset, and in figures \ref{ovrdens1fow}-\ref{ovrdens9fow}, we can see how this evolves.  

\begin{figure}[h]
    \centering
    \begin{minipage}{0.4\textwidth}
        \centering
        \includegraphics[scale=0.3]{figures/fow1density.png}
        \caption{Density of all teams for first wicket falling}
        \label{alldens1fow}
    \end{minipage}
    \begin{minipage}{0.4\textwidth}
        \centering
        \includegraphics[scale=0.3]{figures/fow1densFull.png}
        \caption{Overall density plot for FOW 1}
        \label{ovrdens1fow}
    \end{minipage}
\end{figure}

\begin{figure}[h]
    \centering
    \begin{minipage}{0.4\textwidth}
        \centering
        \includegraphics[scale=0.3]{figures/fow1density.png}
        \caption{Density of all teams for first wicket falling}
        \label{alldens5fow}
    \end{minipage}
    \begin{minipage}{0.4\textwidth}
        \centering
        \includegraphics[scale=0.3]{figures/fow5densFull.png}
        \caption{Overall density plot for FOW 5}
        \label{ovrdens5fow}
    \end{minipage}
\end{figure}

\begin{figure}[h]
    \centering
    \begin{minipage}{0.4\textwidth}
        \centering
        \includegraphics[scale=0.3]{figures/fow9density.png}
        \caption{Density of all teams for first wicket falling}
        \label{alldens9fow}
    \end{minipage}
    \begin{minipage}{0.4\textwidth}
        \centering
        \includegraphics[scale=0.3]{figures/fow9densfull.png}
        \caption{Overall density plot for FOW 9}
        \label{ovrdens9fow}
    \end{minipage}
\end{figure}

In \ref{ovrdens1fow}, we see the density is heavily skewed to the left. This makes sense, as the bowling team will presumably be starting 
their innings by using their best bowlers, who will be hunting to get wickets early on. In \ref{ovrdens5fow}, we see a much more normally distributed
density function. But in actual fact, we see this interesting second, smaller peak appearing lower down in the score. Does this make sense? It's certainly 
not suprising. What these two peaks exemplify is the fact games can go heavily in favour of the bowling team, which can be seen in the first small peak,
wherein they have taken a lot of wickets in quick succession, meaning the later order batters are coming in earlier than usual. Secondly, it shows when the 
batting team is having a good day, because we have this much larger peak around the 200 runs mark.\\

Finally, in \ref{ovrdens9fow}, we can see that the earlier bowlingadvantage peak is much higer, because the lower order batters are traditionally less skilled 
at batting, and so the bowling team have a distinct advantage in taking wickets against these players. But we also see the second, batting-favoured peak is no much lower.
This corresponds to the scenario in which the earlier batters have laid a good foundation of the game, and the lower-order batters have not had to contribute much to the score.

\chapter{Overview of Pattern Recognition Techniques}

In this chapter, we will discuss the mathematical foundations of the patern recognition methods that will be used to predict 
scores. There are two approaches to this. The first is to use already built packages for both a given programming language and simply
feed data into it and present the results. The second method is to build them from scratch. The disadvantage to doing this is naturally 
that it is more time consuming. However, it does allow for more control over how our models work, and will allow us to understand our results
better, than if we had used a proprietary model. This isn't to say that the code written for these models won't utilise packages for doing things
like linear algebra calculations, but we wont simply import a neural network package and have results within 5 lines of code. \\

%Explain which langauges were chosen for each method and why


\section{Neural Networks}

\subsection{A Brief Introduction to Neural Networks}
Neural networks (NNs) have been the subject to a lot of hype in recent years. They are a machine learning method that is being applied to many problems
in all sorts of fields.  %some references here please 
The network will be trained on runrate data, so for each game we have calculated the evolution of the runrate,
and then we have the overall score for that game in the final column of the matrix. An example of a network can be seen in \ref{nnexample1}. The first layer is the 
input layer, and the last layer gives the predictions. The middle layers are hidden and where the work of the network is done. \\

Let's look at what a NN actually looks like. The below is an example of a network with just one hidden layer.

\begin{figure}
    \centering
    \includegraphics[scale=0.5]{figures/nn.png}
    \caption{Example of a single hidden layer neural network, as in \cite{sprbk}}
    \label{nnexample1}
\end{figure}

We see there are $p$ input nodes at the bottom, M nodes in the hidden layer, and K output layers. The lines between the layers are given a ``weight''
and a ``bias''. These properties will be discussed in more detail shortly. The number of hidden layers to be used will be the subject of experimentation. In order to find try this out, we will have to randomly
split the matrix into a training matrix and a testing matrix. We can then evaluate the network by trying to predict values from the testing set and seeing
how well it does. This will then lead to us refining the number of hidden layers as appropriate. \\

One area of the world where neural networks are being applied to value prediction is in the stock market. Naturally if one can predict how the value of a stock
will change over some time period, then one can protect themself from a bad investment, or profit heavily from a good one. We will use similar
methods here for building our neural network.  In \cite{nnstock}, the authors twest two different Neural Netowrks, and find that using a ``Multi-Layer Feed Forward Nerual Network''
is the better choice for predicting how stock values will change. With these motivations in mind, we can begin to construct the networks.

\subsection{Builiding The Network}
%Rewrite this bit
Our input layer will have 50 nodes, one for the runrate at the end of each over. We will begin with 5 hidden layers, although this is subject to change. The output layer
will of course only have one node, the value of which will predict the score of the game. 

We begin by looking at a single node. The proper name for each node is ``perceptron''. Each perceptron takes in the values of a vector, and an extra ``+1'' intercept term. The perceptron then outputs a valye h given by \ref{percepout}:

\begin{equation}
    \label{percepout}
    h_{W,b}(x) = f(\textbf{W}^Tx) = f(\sum_{i=1}^KW_ix_i+b).
\end{equation}

Here, $K \in \mathbb{N}$ is the number of elements in the vector, and $f:\mathbb{R} \rightarrow \mathbb{R}$ is called the activation function. There is a bit of choice in which activation function to use.
The two common choices are:

\begin{align}
    f(z) &= \frac{1}{1+exp(-z)} \\
    f(z) &= tanh(z)
\end{align}

The comparrison of these two functions can be seen in \ref{actfig}.

\begin{figure}[h]
    \centering
    \label{actfig}
    \includegraphics[scale=0.5]{figures/actfuncs.png}
    \caption{Graph showing the shape of different activation functions.}
\end{figure}

We will be using (4.3) as our activation but will return to the activation function shortly. For now, lets look at the network we're going to use.
The initial neural network that we're building. The code for this activation can be seen in \ref{pyact}.

\begin{figure} %MAKE SURE THESE ARE THE RIGHT LINES OF CODE!!
    \lstinputlisting[language=Python, firstline=14,lastline=15]{../code/net.py}
    \caption{Python code for the activation function}
    \label{pyact}
\end{figure}

%Put a picture of the network here when you make it

We begin with getting from the input layer, denoted $L_0$. Each node in $L_0$ takes a value $a^{(0)_i}$ for $i \in \{1,2,\ldots,50\}$. We use $textbf{a}^(0)$ to denote the vector
containing these values. For every node in $L_0$, we have 25 connections coming away, one going to each of the nodes in $L_1$. 25 was an arbitraty choice for the size of $L_1$, and is
subject to change based on results from initial testing. So that means we have $50 \times 25 = 1250$ weights for just the first layer alone. Some linear algebra is to be done here to get 
values for the vector $\textbf{a}^{(1)}$. We use $w_ij$ to denote the weight from node j in one layer to node i in the prior layer. We can then construct a matrix containing the weights
between $L_0$ and $L_1$, which we denote $W^{(1)}$. This matrix is given by \ref{weights1}.

\begin{equation}
    W^{(1)} =
    \left[ {\begin{array}{cccc}
      w_{1,1} & w_{1,2} & \cdots & a_{1,50}\\
      w_{2,1} & w_{2,2} & \cdots & a_{2n}\\
      \vdots & \vdots & \ddots & \vdots\\
      w_{25,1} & a_{25,2} & \cdots & a_{25,50}\\
    \end{array} } \right]
    \label{weights1}
\end{equation}

Using \ref{weights1}, and combining with $a^{(0)}$, we can get the following:

\begin{align}
    A^{(1)} &= \left[ {\begin{array}{cccc}
        w_{1,1} & w_{1,2} & \cdots & a_{1,50}\\
        w_{2,1} & w_{2,2} & \cdots & a_{2n}\\
        \vdots & \vdots & \ddots & \vdots\\
        w_{25,1} & a_{25,2} & \cdots & a_{25,50}\\
      \end{array} } \right] \times \left[ \begin{array}{c}
          a^{(0)}_1 \\
          a^{(0)}_2 \\
          \vdots \\
          a^{(0)}_{50} \\
      \end{array} \right] + \left[ \begin{array}{c}
        b^{(0)}_1 \\
        b^{(0)}_2 \\
        \vdots \\
        b^{(0)}_{25} \\
        \end{array} \right] \\
      &= W^{(1)}\textbf{a}^{0} + \textbf{b}^1
\end{align}

Where $\textbf{b}^1$ is the vector containing the biases for each node.  At this point, we are incredibly close to having values for $\textbf{a}^{(1)}$. The last thing to do
is apply the activation function. The above has resulted in a $25\times 1$ vector, we use the notation $f(A^{(1)})$ to denote applying the activation function to each 
element in the vector $A^{(1)}$. Putting this all together, we have the following equation:

\begin{equation}
    \textbf{a}^{(1)} = f(W^{(1)}\textbf{a}^{(0)} + \textbf{b}^{1})
\end{equation}

Which gives us values for all the nodes in the first hidden layer. This process is then repeated for all the the remaining layers in the network. Implementing this in Python 
is not too difficult, and the code can be seen in \ref{pyActVals}. The only function that can be seen here with no code is $\textit{is\_Valid()}$, which just checks that all the 
constituents are of the right dimensions. If this returns false, then it wont let the calculations go ahead. 

\begin{figure}[h]%AGAIN, MAKE SURE THESE ARE THE RIGHT LINES OF CODE!!
    \lstinputlisting[language=Python, firstline=32,lastline=48]{../code/net.py}
    \caption{Python code for giving activation values to the next layer}
    \label{pyActVals}
\end{figure}


This process is then repeated for each layer, creating a diffferent weights and bias matrix going between each layer in the network. These matrices are naturally of different sizes, but the principle is exactly the same.
When we start the training process for this network, we will only have random variables for the weights and biases. To gain better values that can predict accurate results in the future,
we need to train the network by feeding it feature vectors and their associated outputs. \\

To allow the netowrk to be trained, we first must define a ``cost function''. The way we do this is to input a vector to the input layer, let the network produce a result, say $y'$, which
initially will be hooribly wrong, and square the difference between this and the true value $y$ for that particluar training vector. Put more precisely,
let \textbf{X} be a feature vector containing 50 elements, and $y$ the corresponding output. Then define the cost function, $C(X,y) := (y'-y)^2$. The lower the value of $C(X,y)$, the 
better the network has done at predicting. The average value of $C(X,y)$ for each X and y in our training set, is then a good measure of the network's performance. \\

The cost function is at the heart of how these networks ``learn''. All we're doing is minimising a cost function, to give matrices of weights and biases that
produce the best output. The algorithm for minimising this cost function is called ``gradient descent'' \cite{cauchy}. Suppose we take every single weight and bias of our network,
and turn it into a giant column vector. 

\begin{example}
    In this first example, we initialise four random weights matrices, four bias vectors, and we run through the process outlined above. We know going into this that 
    the cost is going to be high. Initialising the weights as random values $w_{ij} \in [-1000,1000]$, we get an initial cost function for our full dataset of a huge
    3098.4. 
\end{example}

We discussed in section \ref{huber} the need to use a Huber Loss function based on our data. Let us now define this function, as in \cite{huber}.

\begin{definition}
    Given an estimation procedure $f$, the \textbf{Huber Loss Function} is given by:
    \begin{equation*}
        L_{\delta}(a) = \begin{cases}
            \frac{1}{2}a^2  \quad \, &\lvert a \rvert \leq \delta, \\
            \delta(\lvert a \rvert -\frac{1}{2}\delta)  \quad\, &\text{otherwise}
        \end{cases}
    \end{equation*}
\end{definition}

Here, $a$ reffers to residuals, which is essentially our cost function. For that, reason, we can rewrite this loss function as 

\begin{equation}
    L_{\delta}(y,y') = \begin{cases}
        \frac{1}{2}(y-y')^2 \quad \, &\lvert y-y' \rvert \leq \delta, \\
        \delta(\lvert y-y' \rvert -\frac{1}{2}\delta) \quad \, &\text{otherwise}
    \end{cases}
\end{equation}

Implementing this function is not difficult, we can do it in a few lines of Python code, as in \ref{hubercode}

\begin{figure}[h]%AGAIN, MAKE SURE THESE ARE THE RIGHT LINES OF CODE!!
    \lstinputlisting[language=Python, firstline=26,lastline=30]{../code/net.py}
    \caption{Python code for giving activation values to the next layer}
    \label{hubercode}
\end{figure}

\chapter{Analysis and Simmulation}

\section{Analysis}
\subsection{Neural Networks}
The first thing to look for is the normality of the errors in the predicted values. If the errors are normally distrubted (and ideally around 0),
then it means that our predictions are sufficiently accurate. We produce a density plot of the errors using the \textit{ggplot2} library in R.

\begin{figure}[h]
    \centering
    \subfloat[\centering Distribution of Errors]{{\includegraphics[width=.4\linewidth]{figures/errDist.png} }}
    \qquad
    \subfloat[\centering Q-Q Plot of Errors]{{\includegraphics[width=.4\linewidth]{figures/errorqq.png} }}
    \caption{Error distribution with Q-Q plot}
    \label{errDistAndQQ}
\end{figure}

As we can see in this figure, there is a general bell curve, but not quite perfect as we have a bump between -0.5 and -0.75, as well as between 0.125 and 0.5. This 
non-normality is reflected in the Q-Q plot of the error. Measuring accuracy of the results is harder for these value prediction networks is harder than in classification 
problems. This is because we can't construct a prediction matrix. We don't expect the network to predict values down to the exact run, this would require a lot more 
data than is available, and a large amount of experimentation. One metric we can therefore use to see how accurate our model is, is to calculate 
the corrolation between the actual results and the predicted results. Using the inbuilt \textit{cor()} function, we obtain a corrolation value of 
$0.9382$. Given how close this is to 1, which would be perfect corrolation, it is fair to say that this method has done well to predict scores. \\

The issue that one may point out here is that this network has been trained on a full-over dataset, and then been tested on a full-over dataset. But the purpose of this 
investigation has been to look at the scenario in which a full game has been completed. So the method is currently ineffective at doing the task it set out to solve. For this reason,
we must come up with a way to 'fill in' the missing overs. The idea for this is to use Monte-Carlo simmulation. We discussed in \ref{exprr} how depending on the stage of the game,
the runrates can be drawn from one of three normal distributions. So for the overs that are missed, we can simply fill in the gaps by drawing a value from the distribution that the missing over falls 
into. 

\subsection{Interlude: Monte-Carlo Simmulation}
Before simmulating cricket games to test our network on, we first find it necessary to delve into the mathematics of the methods used to for doing the simmulating. This is where
``Monte-Carlo'' simmulating comes in. \\
Let H be some random variable. At this stage, the distribution of H is irrelevant, but we note that $\mu = E(H)$. Formally, we have the following definition.

\begin{definition}
    Let $n \in \mathbb{N}$ and let $\{H_i \ | \ i =1,\ldots,n\}$ be i.i.d copies of H. The \textbf{Monte-Carlo Estimate} of $\mu$, is given as $\hat{\mu}=\frac{1}{n}\sum_{i=1}^n H_i$.  
\end{definition}

Recall the well-known \textit{Strong Law of Large Numbers}.

\begin{theorem}
    \label{slln}
    Let $n \in \mathbb{N}$, and let $H_1$,$H_2$,... be an i.i.d sample from a distribution with expectation $\mu$ and standard deviation $\sigma$, with $\mu, \ \sigma < \infty$.
    Then 
    $$ 
        P\left( \lim_{n \to \infty} \bar{H}_n = \mu \right) = 1
    $$
\end{theorem}

It follows from Theorem \ref{slln} that

\begin{align}
\lim_{n \to \infty} \hat{\mu} &= \lim_{n \to \infty} \frac{1}{n}\sum_{i=1}^nH_i \\
                              &= E(H) \\
                              &= \mu.
\end{align}

\subsection{Match Simulation}
With the theoretical framework for Monte-Carlo simmulation established, we can now look to build an algorithm for simmulating cricket matches. Based on our own work in Chapter 4, 
we begin by defining three random variables, $R_{\text{powerplay}} \sim N(\mu_{\text{powerplay}},\sigma_\text{powerplay})$, $R_{\text{middle}} \sim N(\mu_{\text{middle}},\sigma_{\text{middle}})$
and $R_{\text{final}} \sim N(\mu_{\text{final}},\sigma_{\text{final}})$.

The numerical values for these are given in the following table.

\begin{table}[h]
    \centering
    \begin{tabular}{c | c | c}
        Numerical Values & $\mu$ & $\sigma$ \\
        \hline
        $R_{powerplay}$ & 1.5298 & 0.4486 \\
        $R_{middle}$ & 1.6541 & 0.3355 \\
        $R_{final}$ & 3.1493 & 1.1349 
    \end{tabular}
    \caption{Numerical values of the parameters used for Monte-Carlo simmulation}
\end{table}

The code for doing this simmulation was not hard to write, and after a few small performance enhancements ran almost instantaneously. The code can be seen in figure \ref{mcecode}.

\begin{figure}[h] %MAKE SURE THESE ARE THE RIGHT LINES OF CODE!!
    \label{mcecode}
    \lstinputlisting[language=R, firstline=5,lastline=25]{../code/PyScripts/monteCarlo.py}
    \caption{Implementing a Monte-Carlo Simmulation fo Cricket Matches}
\end{figure}

It is best to think of n as a sort of fine-tuning parameter. If we have n too large, we would simply be simmulating the average cricket game, while have n too low and 
we are open to having an outlier. As for the other parameters, we have \textit{overs} being the number of overs in the game, and \textit{state} determines the over to start 
simmulating from. So if we want to simmulate the whole game, set $state=1$ and $overs=50$. The array \textit{runrates} just holds the data.\\

We then have the function \textit{MCE(n,state)}. The first line of this is a ``Ternary Operator'' to determine a parameter $t$. The job of $t$ is to be an index which obtains 
the correct parameters from the arrays in lines 2 and 3. This is then fed into the next line, uses the \textit{random.normalvariate()} function to populate an empty list with random 
values drawn fro the appropriate distribution of R. This is then averaged and returned by the function. Completing the main part of the Monte-Carlo method.
Finally, a while-loop runs through the overs needed and adds the estimation values to the \textit{runrates} array. \\

To give an idea of how the value of $n$ affects the resulting simmulation, we ran two simulations, using a small value $n=5$, and a larger one using $n=100$. 

\begin{figure}[h]
    \centering
    \subfloat[\centering $n=5$]{{\includegraphics[width=.4\linewidth]{figures/mcen5.png} }}
    \qquad
    \subfloat[\centering $n=100$]{{\includegraphics[width=.4\linewidth]{figures/mcen100.png} }}
    \caption{50-Over simmulation of $n=5$ and $n=100$}
    \label{MeanAndSDRR}
\end{figure}

If we compare these figues with (a) in Figure \ref{MeanAndSDRR}, we see that with large $n$, we have fallen victim to the ``Central-Limit Theorem'', and it looks as if the three 
sections of the game are unrelated. In the end, $n=20$ was the chosen value as it provided an appropriate middleground. 

\chapter{Simmulating the ICC 2019 Cricket World Cup}
In this chapter, we use the model we have built and apply it to games decided by DLS in the 2019 cricket world cup. The objective is to see if our model 
gives a similar outcome to the tournament as DLS did, or to see how the tournament would have differed using our model. WE give a sceheme for how score targets will
be reset, and use this to simmulate the games in question. 

\section{DLS in the 2019 World Cup}
Only three games were decided by DLS in the 2019 Worlc Cup, two of which involved Afghanistan. The other was a game between India and Pakistan. The ball-by-ball data for 
each of these games was collected and tidied in the previous ways. There are three different ways in which we can apply the neural network depending on how much of the first innings gets played.

\begin{itemize}
    \item If the full first innings isn't played, predict a score using the network, and give the target as the proportional score to the number of overs available. 
    \item If the full first innings is played, and some of the second innings is played, apply the network to the equivalent overs in the first innings, and use that to set the score. 
\end{itemize}


\section{Applying the Neural Network Model}
\subsection{Sri Lanka vs Afghanistan}
Batting first, Sri-Lanka ended their rain-affected innings on 201 after 36.5\footnote{NOte that in cricketing notation, this means 36 overs and 5 balls, not 36 overs and 3 balls, which would be the case if .5 meant half an over.} overs.
Under DLS, Afghanistan were set a target of 187 from 41 overs. However, Afghanistan were bowled out for 152 from 32.4 overs, coming up 34 runs short of the required target from just 79.7\% of their alloted innings. 
Using the data from the first Innings, our model predicted that Sri Lanka would have gone on to score 338 runs had they have had a full innings. Since Sri Lanka only played $61.3\%$ of their innings, under this method, Afghanistan 
would have been set a target of $0.613 \times 338 = 207 \text{ runs}$. However, that is if Afghanistan were given a full 50 overs, which they didn't have in this game due to light. For that reason, we need 
to again reduce this by the ration of available overs, 41 in this case. Therefore we reduce this by $\frac{41}{50}=0.94$, giving a reduced target score of $195$ runs from 41 overs. This is considerably higher than the 
score set by DLS, and given how Afghanistan batted, quite far out of reach. This means our method has had no affect on the standings of the World Cup from this game.

\subsection{Afghanistan vs South Africa}
This time batting first, Afghanistan were bowled out for 125 from 34.1 overs. Midway through the first innings, the game was reduced to 48 overs per side. After the first innings, South Africa were set 
a target of 127 from their 48 overs. Interestingly, our model predicted that Afghanistan would reach a score of 125 had they have had a full 50 overs. Strictly speaking, in this scenario that's a perfect prediction 
as they had no batters left so couldn't go any higher than the 125 they actually did achieve! Since South Africa had 48 overs available to them, that's an increase of 1.4, so the score target set under our scheme
is $1.4 \times 125 = 175$ runs from 48 overs. Using our model on the South Africa innings, a predicted score of 151 is obtained. From 48 overs instead of 50 this gives a predicted score of $0.96 \times 151 = 145$. 
This naturally falls up short of the 175 target, so actually Afghanistan win this game, and would give them their only win of the tournament. They stay bottom of the points table, South Africa however would have 
dropped two points and fallen to eight place, allowing Bangladesh to move up into position seven. 

\subsection{India VS Pakistan}
After a stellar innings from Indian batsman Rohit Sharma, hitting 140 off 113 deliveries, India finished their 50 overs with a score of 336- losing only 5 wickets in the process. Pakistan stepped up to the plate, reaching 
166 from 35 overs, chasing 337. But then the rain came down over Manchester and the score was revised by DLS to them needing 136 more runs from just 5 overs. This requires a mammouth effort of 27.20 runs per over. Needless to say,
Pakistan fell short, going on to hit a total of 212 from their 40 overs. Our model predicted that after the $35^{th}$ over, India would go on to score 284 in their 50. Reducing this by 0.2 to give a predicted score after 40 overs, 
we get a target score of 227. Paksitan reached 212 off their 40, so fall short and still lose, meaning nothing changes in the table. Given that Pakistan were on 166, this new target of 61 from 5 overs is far more realistic, and would have 
meant for a far more enjoyable game for the spectators, as it would have at least given Pakistan a fighting chance, and not essentially killed the game by setting such a high target from a short number 
of overs. 

\section{Summary of Model affect on the Tournament}
There is only one change to the final table, and that change has no impact on the overall impact of the tournament, since India still won against Pakistan. Afghanistan and South Africa both finished in the bottom 
half of the table, and even with our method, South Africa would have only dropped one place, while Afghanistan's position wouldn't change. \\

While our model gave similar outcomes to DLS, the score targets set were arguably more realistic in some cases, especially the India-Pakistan game, and certainly from an entertainment perspective, this 
has great value in maintaining the competaive aspect of the game. Naturally, using proportions as a way to decrease or increase something that is inherently random isn't ideal, but this leaves room for future work 
on the topic. 

\chapter{Conclusions and Future Work}

\epigraph{I'm completely cricketed out. If I never have to write another word about cricket again, I'll be a happy man.}{Joseph O'Neill}

\section{Conclusions}




\section{Discussion and Future Work}
Following on from what was mentioned in Section~\ref{mcsim}, rather than simply decreasing scores in proportion with the number of overs loft, 
it would be worth investigating decreasing the score depending on the fall of wicket distribution, as we looked at earlier in the paper. This could be 
done using Bayesian Techniques. 

\appendix

\chapter{Q-Q Plots and the Normal Distribution}

Q-Q plots have been used extensively throughout this project, due to their utility in testing if a sample is normally distributed.
For that reason, delving into the maths behind them a bit more helps with interpreting their results. 

At its core, a Q-Q plot shows the quantiles of two distributions against one another. This can either be drawn from two exact datasets,
or, as is common in our case, one dataset against samples from a particular probability distribution. This is the case this appendix will focus on. 
The Q in Q-Q plot refers to ``Quantile''. Quantile functions rely on the distribution function of a particular distribution.

\begin{definition}
    Suppose X is a random variable. The \textbf{Cumulative Distribution Function} (CDF) of X is $F:\mathbb{R} \to [0,1] $ given by
    \[
        F(x) = P(X \leq x)  
    \]
\end{definition}

In the case of the normal distribution, we have the following lemma.

\begin{lemma}
    Let X be normally distributed with mean $\mu$ and variance $\sigma^2$. Then the cumulative distribution function of X is 
    \[
        F(x) = \Phi\left(\frac{x-\mu}{\sigma}\right) = \frac{1}{\sigma \sqrt{2 \pi}}\int_{\infty}^{x}exp\left( -\frac{(t-\mu)^2}{2\sigma^2} \right) dt
    \]
\end{lemma}

In almost every application throught this project, we have been looking to see if data follows a standard normal distribution. In which case, the
CDF is 

\[
    F(x) = \Phi(x) = \frac{1}{\sqrt{2\pi}}\int_{\infty}^{x}exp\left( -\frac{t^2}{2}\right).  
\]

The following lemma will allow us to use a useful result in building Q-Q plots.

\begin{lemma}
    $\Phi(x)$ is monotonically strictly increasing
\end{lemma}

\begin{proof}
    We have the standard result $\frac{d}{dx}\left(exp(-x)\right)$ = $-exp(-x)$. Now since $exp(x)$ is by definition strictly increasing, $exp(-x)$ is strictly decreasing. It 
    therefore follows that $-exp(-x)$ is strictly increasing. From the fact $\frac{1}{\sqrt{2\pi}}$, and the prior argument, it follows that $\Phi(x)$ is strictly increasing.
    The fact that this is the case on the entire domain of $exp(x)$, means that $\Phi(x)$ is also monotonic. 
\end{proof}

The reason we need to show $\Phi(x)$ fits this in particular property is that if for some distribution with CDF $F$, $F$ is continuous and strictly monotonically increasing, then 
the \textit{Quantile Function} $Q$ is given by $Q = F^{-1}$. Infact, the standard normal distribution's quantile function, $\Phi^{-1}(p)$ $p \in (0,1)$, is called the \textit{Probit}
function. The Probit function is defined in terms of the relative error function. Formally, this is given by the following:

\begin{definition}
    The \textbf{Relative Error Function}, $erf(x)$, gives the probability that the random variable $X \sim N(0,\frac{1}{2})$ takes a value between $-x$ and $x$ inclusive. 
    \[
        \text{erf}(x) = \frac{2}{\sqrt{\pi}}\int_0^x exp(-t^2)dt.    
    \]
\end{definition}

With that, we can formally define the quantile function of the standard normal distribution.

\begin{definition}
    The \textbf{Probit} function, $probit(x)$ is given by:
    \[
        probit(x) = \sqrt{2}\text{erf}^{-1}(2p-1).  
    \]
\end{definition}

It is this probit function that forms the x-axis in our Q-Q plots. The y-axis contains the sample quantiles. Consider the case where we sampled the same (theoretical) distribution twice,
putting one on the y-axis and one on the x-axis. Clearly, the plot would form a straight line equivalent to $f(x)=x$ on the 2D cartesian grid. This is the rationale behind the utility of the plot, since 
if the line is as close to $f(x)=x$ as possible, then we can say with a fairly high confidence that the sample distibution follows the theoretical one we are testing against. If, however, the line formed is curved or differs in another way, then it is safe to say the sample distribution doesn't follow that distribution. 

\chapter{Neural Network Plot}

The Neural Network produced in Chapter 5 is quite large, but the \verb|neuralnet| package does allow us to view it nonetheless. \\

\begin{figure}[h] 
    \centering
    \includegraphics[width=1.2\linewidth]{figures/bestnet.png}
    \caption{Final trained network. }
    \label{bestnet}
\end{figure}

It is hard to see the exact weights and biases themselves, but these values aren't individually that important, the plot is more for 
demonstrative purposes.

\chapter{The CricNet Package}

This chapter contains no new mathematics or results. We discuss turning the code written here into an R package that can be used for easily expanding on the work of this project.
The package comes complete with the dataset that we trained our network one, meaning the interested reader could replicate our results or even improve upon them. 

\section{Introduction}

R as a language is built on packages, the \textbf{C}omprehensive \textbf{R} \textbf{A}rchive \textbf{N}etwork, CRAN, stores thousands of packages that can be used for performing tasks in R. This saves people re-inventing the wheel when performing statistical analyses. In this project, we have used CRAN to obtain 
access to the several packages that were imperitive in undertaking the task of building a neural network. 
Packages are installed from CRAN using the function \textit{install.packages(``package'')} and loaded with \textit{library(``package'')}. However, due to the relatively small scale of this project, we won't be storing the package on CRAN (at this point in time), but rather on Github. This is preferable due to how easy it is to access code on Github, and because it also allows other people to directly contribute to the expansion of the package should they wish.  

\section{Building an R Package}

Building an R package is quite a simple concept- we are just putting all our data and functions into one place that can allow for easy modification and the replicating of results. Actually building a package requires using other packages, specifically the \verb|devtools| package. Running the function \textit{usethis::create\_package()} from the \verb|usethis| package creates a directory with the necessary files for an R package. This file structire initially can be seen in the directory tree below. \\

\begin{figure}[h]
\caption{Default File Structure for an R Package}

    \dirtree{%
    .1 CricNet.
    .2 CricNet.Rproj.
    .2 DESCRIPTION.
    .2 NAMESPACE.
    .2 R.
    .2 data.
    }    



\end{figure}

The R directory is unsurprisingly where the R files containing all the functions go. Naturally not every piece of code we wrote will go into the package, as some of it was simply for demonstrative purposes. All the code for actually running the network (which is more or less a wrapper on the \verb|neuralnet| package), along with the code for analysing those results goes into the package.
Each function gets its own file, for ease of access, documentation and debugging. \\

The files NAMESPACE and DESCRIPTION are metadata files used by the \verb|roxygen2| package for mainting things such as version number, liscencing and name(s) of the author(s). NAMESPACE is also used to store any dependencies on packages that the package relies on. For example, our \textit{scoreNet.R} file, as mentioned, is a wrapper on the function \textit{neuralnet::nerualnet()}, so it clearly needs to import the required package.


\section{CricNet Structure}

The documentation for what each function does can be found within the package documentation, or using \textit{help(function)} in the R console. The purpose of this section is to give an overview of the package and how to use to replicate results from this project. \\

The CSV containg the run rates that the network was initially trained on exits in the \textit{data} directory as ``rrmat.rda''. Saving this data is done using the function \textit{usethis::use\_data()} function which puts everything in the right format and direcotry automatically. There are four R functions that come with the package, ``scoreNet.R'', ``genResults.R'', ``unscale.R'' and ``netAnalysis.R''. Note that they should be used in this order. No output comes from the first two, but by running them, several R objects are created. One for the network itself (an object of type \textit{nn}), and one for storing the results. The analysis script takes this results dataframe and displays a corrolation score and a density plot of the errors. In the future, more analysis features will be added to this function. Release version 0.1 of the package is the one that corresponds directly to the work of this project, with no expansions.\\

Adding documentation to each function is done using the \verb|roxygen2| package. It is a case of simply writing docstrings above the functions in their respective files, and using \textit{usethis::document()} to build the file. In the docstring, the ``export'' tag is what allows the functions to be used directly by anyone who installs the package.  
These pieces of documentation create new ``.Rd'' files in the ``/man'' directory. Storing them here allows them to be viewed alone by running the command \textit{help(``function'')} or \textit{?``function''} in an R console. 

The overall structure of the \verb|CricNet| package can be seen in the directory tree in Figure~\ref{cricnetstruct}.\\

\begin{figure}[ht]
\caption{File structure for the CricNet package}

    
\dirtree{%
.1 CricNet.
.2 CricNet.Rproj.
.2 data.
.3 rrmat.rda.
.2 DESCRIPTION.
.2 man.
.3 genResults.Rd.
.3 netAnalysis.Rd.
.3 scoreNet.Rd.
.3 unscale.Rd.
.2 NAMESPACE.
.2 R.
.3 genResults.R.
.3 netAnalysis.R.
.3 scoreNet.R.
.3 unscale.R.
.2 readme.md.
}

\label{cricnetstruct}
\end{figure}

\section{Using The Package}
The package can be accessed at ``https://github.com/mattknowles314/CricNet''. It can be installed directly in R using the \verb|devtools| function \textit{install\_github(mattknowles314/CricNet)}. Any bugs or issues can be reported via the ``Issues'' tab on the Github page. 


\bibliographystyle{apalike}
\bibliography{refs}{}


\end{document}
